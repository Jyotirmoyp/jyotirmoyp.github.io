\documentclass[11pt,a4paper]{article}
\usepackage[utf8]{inputenc}
\usepackage[english]{babel}
\usepackage{amsmath}
\usepackage{amsfonts}
\usepackage{amssymb}
\usepackage{makeidx}
\usepackage{graphicx}
\usepackage{enumitem}
\usepackage{lmodern}
\usepackage{longtable}
\pagestyle{empty}
\usepackage{hyperref}
\hypersetup{
    colorlinks=true,
    linkcolor=blue,
    filecolor=magenta,      
    urlcolor=blue,
%    pdftitle={Overleaf Example},
%    pdfpagemode=FullScreen,
    }

%\usepackage{kpfonts}
\usepackage[left=2cm,right=2cm,top=2cm,bottom=2cm]{geometry}
\usepackage{fontawesome5}

\usepackage{longtable}
%\usepackage{polyglossia}
%\setmainlanguage[numerals=Devanagari]{bengali}
%\setotherlanguage{english}
%\newfontfamily\englishfont[Scale=MatchLowercase]{Linux Biolinum O}
%\newfontfamily\bengalifont[Script=Bengali]{Akaash}

%\usepackage[banglamainfont=Kalpurush,
%banglattfont=Siyam Rupali
%]{latexbangla}

\begin{document}

\begin{center}
\textbf{{\Large JYOTIRMOY PAUL}}\\
%Curriculum Vitae\\
Centre for Planetary Habitability, University of Oslo\\
\faEnvelope \quad \href{mailto:jyotirmoy.paul@geo.uio.no}{jyotirmoy.paul@geo.uio.no}\\
\faIcon{globe} \href{https://jyotirmoyp.github.io}{jyotirmoyp.github.io} \quad
\href{https://orcid.org/0000-0002-7972-1868}{\faIcon{orcid}} \quad %\href{https://twitter.com/GeophyJo}{\faIcon{twitter}} 
\quad \href{https://github.com/Jyotirmoyp}{\faIcon{github}}
\end{center}

\vspace{-0.3 in}
\begin{longtable}{p{5cm} p{10 cm}}
    


\hline
\textbf{APPOINTMENTS} &  \\
\hline 
2024-current & \textbf{Marie Skłodowska–Curie Fellow}, University of Oslo \\
2021 Nov - 2021  & \textbf{Post Doc}, Bayerisches Geoinstitut, University of Bayreuth \\

2021 July - Oct & \textbf{Research Associate}, Indian Institute of Science \\
\hline


%\vspace{0.02 in}
\textbf{EDUCATION} &  \\

\hline
2015 - 2021 & \textbf{PhD}, Indian Institute of Science, Bangalore\\

2013 - 2015 & \textbf{Master of Science (M.Sc)}, Jadavpur University
\\

2010 - 2013 & \textbf{Bachelor of Science (B.Sc.)}, Jadavpur University


\end{longtable}
\vspace{-0.5 in}
\begin{longtable}{p{10 cm} p{5 cm}}
\hline
\textbf{AWARDS}& \\
\hline
\end{longtable}
\vspace{-0.3 in}
\begin{enumerate}[noitemsep]
\item 
Marie Skłodowska–Curie fellowship, 2024-2027
\item Independent researcher travel grant, University of Bayreuth, 2022
\item American Geophysical Union travel grant, 2021, 2019, 2018
\item Roland Schlich grant for early career scientists by European Geoscience Union, 2018, 2021
\item Tata trust grant, Indian Institute of Science, Bangalore, 2019
\item Augmenting Writing Skills for Articulating Research, DST, India, 2019
\item INSPIRE Scholarship, DST, 2010-2015
\end{enumerate}

\vspace{-0.3 in}
\begin{longtable}{p{10cm} p{5cm}}
\hline
 \textbf{ACADEMIC ACHIEVEMENTS} & \\
\hline
\end{longtable}
%\begin{longtable}{p{1 cm} p{14 cm}}
\vspace{-0.3 in}
 \begin{enumerate}[noitemsep]
\item \textbf{Highest grade in PhD course work }(7.1/8) 
\item \textbf{Second highest marks} in Postgraduate batch 2013-15 (82.25\%) 
\item \textbf{Third highest marks} in Undergraduate Batch 2010-13 (81.7\%) 
\item \textbf{Gold medalist} for highest marks in undergraduate $1^{st}$ and $2^{nd}$ year
\item \textbf{National Eligibility Test} (NET) : Rank-18 (2015), 99 (2014)
\item \textbf{Graduate Aptitude Test in Engineering} (GATE): Rank-212 (2015)
\item \textbf{First} in the Presidency College Geology undergraduate admission test, 2010
\end{enumerate}
\vspace{-.3 in}
\begin{longtable}{p{5cm} p{10cm}}
\hline
 \textbf{COMMUNITY WORK} & \\
 \hline
\end{longtable}
\vspace{-0.2 in}

\begin{enumerate}[noitemsep]
\vspace{-0.1 in}
\item \textbf{Outstanding Student Poster Presentation contest coordinator} (2023-current), \textbf{Blog editor} (2020-2021), \textbf{Social media co-ordinator} (2019-2021) for \textit{EGU Geodynamics}

\item \textbf{Reviewer}: \textit{Nature Geoscience, Nature communications: Earth and Environment, Physics of the Earth and Planetary Interiors, Geophysical Journal International, National Science Foundation (USA), Lithosphere}

\item \textbf{Conference organisation}: Session chair at AGU: 2019, 2022

\item \textbf{Guest-photographer \& Social media co-ordinator}: AGU: 2018-2019
\end{enumerate}  

\vspace{-0.3 in}
\begin{longtable}{p{10cm} p{5 cm}}
\hline
\textbf{ACADEMIC VISITS and INVITED TALKS} &  \\
\hline
\end{longtable}
\vspace{-0.3 in}
\begin{enumerate}[noitemsep]
\item[•] \textbf{Visits}:  ETH Zürich (2022); IIT Bombay (2012, 2013); Schlumberger Mumbai (2014)
\item[•] \textbf{Invited Talks}: Arizona state, Durgrapur Govt. (2024); AGU (2023); German-Swiss Geodynamics workshop (2023); IIT Gandhinagar (2023); University of Oslo (2022); etc.
\end{enumerate}

\vspace{-0.3 in}
\begin{longtable}{p{7cm} p{8 cm}}
\hline
\textbf{RECENT PUBLICATIONS} & \href{https://jyotirmoyp.github.io/}{Full list:  jyotirmoyp.github.io} \\
\hline
\end{longtable}
\vspace{-0.3 in}
\begin{enumerate}[noitemsep]
\item J. Paul*, G. J. Golabek, A. B. Rozel, P. J. Tackley, T. Katsura and H. Fei (2024). Effect of bridgmanite-ferropericlase grain size evolution on Earth's average mantle viscosity: Implications for mantle convection in early and present-day Earth. Progress in Earth and Planetary Sciences, 11 (64) \href{https://doi.org/10.1186/s40645-024-00658-3}{[doi]}.
\item J.Paul*, C.P. Conrad, T.W. Becker, A. Ghosh, 2023. Convective craton self-compression and its role for  stabilization of old lithosphere, \textit{Geophysical Research Letters}, e2022GL101842 \href{doi:10.1029/2022GL101842}{[doi]}.

\item J. Paul*, A. Ghosh, 2022. Could the Reunion plume have thinned the Indian craton?, \textit{Geology},  v. 50, p. 346–350 \href{https://doi.org/10.1130/G49492.1}{[doi]}.


\item[in prep. ] J.Paul*, A. Spang, A. Piccolo. Flat slab induced weakening and destruction of the North China craton.


\end{enumerate}



\end{document}